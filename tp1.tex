\documentclass[10pt, a4paper]{article}
\usepackage[paper=a4paper, left=1.5cm, right=1.5cm, bottom=1.5cm, top=3.5cm]{geometry}
\usepackage[latin1]{inputenc}
\usepackage[T1]{fontenc}
\usepackage[spanish]{babel}
\usepackage{indentfirst}
\usepackage{fancyhdr}
\usepackage{latexsym}
\usepackage{lastpage}
\usepackage{aed2-symb,aed2-itef,aed2-tad}
\usepackage[colorlinks=true, linkcolor=blue]{hyperref}
\usepackage{calc}

\newcommand{\f}[1]{\text{#1}}
\renewcommand{\paratodo}[2]{$\forall~#2$: #1}


\sloppy

\hypersetup{%
 % Para que el PDF se abra a página completa.
 pdfstartview= {FitH \hypercalcbp{\paperheight-\topmargin-1in-\headheight}},
 pdfauthor={Grupo MISSINGNO - Algoritmos y Estructuras de Datos II},
 pdfkeywords={Trabajo Practico 1},
 pdfsubject={Alta seguridad nos cuida}
}

\parskip=5pt % 10pt es el tamaño de fuente

% Pongo en 0 la distancia extra entre ítemes.
\let\olditemize\itemize
\def\itemize{\olditemize\itemsep=0pt}

% Acomodo fancyhdr.
\pagestyle{fancy}
\thispagestyle{fancy}
\addtolength{\headheight}{1pt}
\lhead{Algoritmos y Estructuras de Datos II}
\rhead{$1^{\mathrm{er}}$ cuatrimestre de 2012}
\cfoot{\thepage /\pageref{LastPage}}
\renewcommand{\footrulewidth}{0.4pt}

\author{Grupo MISSINGNO - Algoritmos y Estructuras de Datos II}
\date{2do cuatrimestre 2015}
\title{Trabajo Practico 1}

\begin{document}

%Pagina de titulo e indice
\thispagestyle{empty}

\maketitle
\tableofcontents

\newpage

\section{TADs Auxiliares}
\par \textbf{TAD} as \textbf{ES} tupla(posicion $pos$, nat $capturas$, nat $sanciones$)
\par \textbf{TAD} posicion \textbf{ES} tupla(nat $x$, nat $y$)
\par \textbf{TAD} movimiento \textbf{ES} tupla(posicion $pos$, direccion $dir$)
\par \textbf{TAD} direccion \textbf{ES} enum\{arriba, abajo, izquierda, derecha\}
\par \textbf{TAD} placa \textbf{ES} nat

%TADS
\section{TAD \tadNombre{Campus}}

\begin{tad}{\tadNombre{Campus}}

\tadGeneros{campus}

\tadExporta{campus, generadores, observadores basicos}

\tadIgualdadObservacional{c1}{c2}{campus}{$(grilla(c1) \igobs grilla(c2))\ \land$\\ $(asshole(c1) \igobs asshole(c2))\ \land$\\ $(estudiantes(c1) \igobs estudiantes(c2)) \ \land$\\ $(hippies(c1) \igobs hippies(c2))$}

\tadAlinearFunciones{posNoRepetida?}{campus/c, posicion/p, direccion/dir}{}

% Generadores
\tadGeneradores

\tadOperacion{crearCampus}{grilla/g, dicc(placa, as)/ass}{campus}{asValidos?(g, ass)}
\tadOperacion{entraEstudiante}{campus/c, posicion/p}{campus}{posValida?(grilla(c), p) $\land$ posVacia?(c, p) $\land$ esEntrada?(c, p)}
\tadOperacion{entraHippie}{campus/c, posicion/p}{campus}{posValida?(grilla(c), p) $\land$ posVacia?(c, p) $\land$ esEntrada?(c, p)}
\tadOperacion{saleEstudiante}{campus/c, posicion/p}{campus}{esEstudiante?(c, p) $\land$ esEntrada?(c, p)}
\tadOperacion{movEstudiante}{campus/c, posicion/ini, posicion/fin}{campus}{esEstudiante?(c, ini) $\land$ posValida?(c, fin) $\land$ posVacia?(c, fin) $\land$ adyacentes?(ini, fin)}
\tadOperacion{movHippie}{campus/c, posicion/p}{campus}{esHippie?(c, p)}
\tadOperacion{movAS}{campus/c, placa/p}{campus}{esAS?(c, p)}

% Observadores
\tadObservadores

\tadOperacion{grilla}{campus}{grilla}{}
\tadOperacion{asshole}{campus}{dicc(placa, as)}{}
\tadOperacion{estudiantes}{campus}{conj(posicion)}{}
\tadOperacion{hippies}{campus}{conj(posicion)}{}

% Otras operaciones
\tadOtrasOperaciones

\tadOperacion{asValidos?}{grilla, dicc(placa, as)}{bool}{}
\tadOperacion{iterarClaves}{grilla, dicc(placa, as), conj(placa)}{bool}{}
\tadOperacion{posValida?}{grilla, posicion}{bool}{}
\tadOperacion{sinCapturas?}{dicc(placa, as)}{bool}{}
\tadOperacion{sinSanciones?}{dicc(placa, as)}{bool}{}
\tadOperacion{posNoRepetida?}{ass, conj(placa), posicion}{bool}{}

\tadOperacion{posVacia?}{campus, posicion}{bool}{}

\tadAlinearFunciones{movEntradaCercana}{posicion, conj(posicion), campus}{}

\tadOperacion{posSinAS?}{campus, posicion}{bool}{}
\tadOperacion{esHippie?}{campus, posicion}{bool}{}
\tadOperacion{esEstudiante?}{campus, posicion}{bool}{}
\tadOperacion{esEntrada?}{campus, posicion}{bool}{}
\tadOperacion{adyacentes?}{posicion, posicion}{bool}{}
\tadOperacion{esAS?}{campus, placa}{bool}{}
\tadOperacion{nuevaPosAS}{campus, placa}{posicion}{}
\tadOperacion{hayHippies?}{campus}{bool}{}
\tadOperacion{movEntradaCercana}{campus, placa}{dicc(placa,as)}{}
\tadOperacion{posAS}{campus, placa}{posicion}{}
\tadOperacion{nuevaPosicion}{posicion, conj(posicion),campus}{posicion}{}
\tadOperacion{entradasCercanas}{posicion,grilla}{conj(posicion)}{}
\tadOperacion{distancia}{nat,nat}{nat}{}
\tadOperacion{seConvierteEnHippie?}{campus, posicion}{bool}{}
\tadOperacion{estudiantesCapturados}{campus, posicion}{conj(posicion)}{}
\tadOperacion{estudiantesFalopeados}{campus, posicion}{conj(posicion)}{}
\tadOperacion{nuevaPosHippie}{campus, posicion}{posicion}{}

\tadOperacion{movYActualizarAS}{placa,posicion,campus}{dicc(placa,as)}{}
\tadOperacion{actualizarAs}{posicion,dicc(placa,as),campus}{dicc(placa,as)}{}

\tadOperacion{hayHippiesAtrapados}{posicion,campus}{bool}{}
\tadOperacion{posConHippies}{conj(posicion),conj(posicion)}{conj(posicion)}{}
\tadOperacion{posContiguas}{posicion,grilla}{conj(posicion)}{}
\tadOperacion{posValidasCont}{conj(posicion),grilla}{conj(posicion)}{}
\tadOperacion{conjuntoPosAs}{conj(placa),dicc(placa,as)}{conj(posicion)}{}
\tadOperacion{revisarHippies}{conj(posicion),campus}{bool}{}
\tadOperacion{esAtrapadoH?}{posicion,conj(posicion),campus}{bool}{}
\tadOperacion{hayPosVacia?}{conj(posicion),campus}{bool}{}
\tadOperacion{hayEstudiantesAtrapados}{posicion,campus}{bool}{}
\tadOperacion{posConEstudiantes}{conj(posicion),conj(posicion)}{conj(posicion)}{}
\tadOperacion{revisarEstudiantes}{conj(estudiantes),campus}{bool}{}
\tadOperacion{esAtrapadoE?}{posicion,conj(posicion),campus}{bool}{}

\tadOperacion{premiarAs}{conj(placa)/placas,campus/c}{dicc(placa,as)}{placas $\subseteq$ claves(asshole(c))}
\tadOperacion{rodeo}{conj(posicion),campus}{conj(posicion)}{}
\tadOperacion{placaDelAs}{posicion/pos,conj(placa)/conjplacas,dicc(placa,as)/dic}{placa}{pos $\in$ conjuntoPosAs(conjplacas,dic)}
\tadOperacion{asParaPremiar}{conj(posicion),campus}{conj(placa)}{}
\tadOperacion{hAtrapados}{conj(posicion),campus}{conj(posicion)}{}

\tadOperacion{capturadoH}{posicion,campus}{bool}{}
\tadOperacion{capturadoE}{posicion,campus}{bool}{}
\tadOperacion{estaEnEntrada?}{posicion,grilla}{bool}{}

\tadOperacion{distanciaTaxista}{posicion,posicion}{nat}{}
\tadOperacion{menorDistanciaAPos}{posicion/pos,con(posicion)/conjunto}{nat}{$\#$conjunto$>0$}
\tadOperacion{personasCercanas}{posicion,conj(posicion)}{conj(posicion)}{}
\tadOperacion{sancionarAs}{conj(placa)/placas,campus/c}{dicc(placa,as)}{placas $\subseteq$ claves(asshole(c))}
\tadOperacion{eAtrapados}{conj(posicion),campus}{conj(posicion)}{}
\tadOperacion{premiarYSancionar}{conj((placa,bool)),campus}{dicc(placa,as)}{}
%ojo con la funcion de arriba, false -> premiar, true-> sancionar

\tadOperacion{perseguirHippies}{placa,campus}{dic(placa,as)}{}
\tadOperacion{hippiesMasCercanos}{posicion,conj(posicion)}{conj(posicion)}{}

%\tadOperacion{posParaMoverse}{posicion,conj(posicion),campus}{conj(posicion)}{}


% Axiomas
\tadAxiomas[\paratodo{campus}{c}, \paratodo{grilla}{g}, \paratodo{posicion}{p}, \paratodo{direccion}{dir}, \paratodo{dicc(placa,as)}{ass}, \paratodo{placa}{id}]
\tadAlinearAxiomas{estudiantes(movEstudiante(c, ini, fin))}

\tadAxioma{grilla(crearCampus(g, ass))}{g}
\tadAxioma{grilla(entraEstudiante(c, p))}{grilla(c)}
\tadAxioma{grilla(entraHippie(c, p))}{grilla(c)}
\tadAxioma{grilla(saleEstudiante(c, p))}{grilla(c)}
\tadAxioma{grilla(movEstudiante(c, p, dir))}{grilla(c)}
\tadAxioma{grilla(movHippie(c, p))}{grilla(c)}
\tadAxioma{grilla(movAS(c, id))}{grilla(c)}

\tadAxioma{asshole(crearCampus(g, ass))}{ass}
\tadAxioma{asshole(entraEstudiante(c, posE))}{actualizarAs(posE,asshole(c),c)}
\tadAxioma{asshole(entraHippie(c, posH))}{actualizarAs(posH,asshole(c),c)}
\tadAxioma{asshole(saleEstudiante(c, posE))}{asshole(c)}
\tadAxioma{asshole(movEstudiante(c, posE, dir))}{actualizarAs(posE,asshole(c),c)}
\tadAxioma{asshole(movHippie(c, posH))}{actualizarAs(posH,asshole(c),c)}
\tadAxioma{asshole(movAS(c, posAs))}{ \IF $\pi_2$(Obtener(posAs,asshole(c))) $\geq3$ 
    THEN asshole(c)
    ELSE {
        \- \IF $\neg$hayHippies?(c) THEN movEntradaCercana(c, p)
        ELSE perseguirHippies(p,c)
        FI
    }
    FI
}

\tadAxioma{posAS(c, p)}{$\pi_1$(Obtener(p,asshole(c)))}

\tadAxioma{hayHippies?(c)}{\IF $\#$Hippies(c) $> 0$ THEN true ELSE false FI}

\tadAxioma{movEntradaCercana(c, p)}{\IF nuevaPosicion(posAS(c, p),posQueAcercan(c,posAS(c,p),entradasCercanas(posAS(c, p),grilla(c)))= posAS(c, p)
    THEN asshole(c)
    ELSE moverseYActualizar(p,nuevaPosicion(posAS(c, p),posQueAcercan(c,posAS(c,p),entradasCercanas(posAS(c, p),grilla(c))),c)
    FI
}

\tadAxioma{nuevaPosicion(posactual,conjpos,c)}{\IF $\emptyset$?(conjpos) THEN posactual
    ELSE {
    \IF estaVaciaPos?(DameUno(conjpos),c) THEN DameUno(conjpos)
        ELSE nuevaPosicion(posactual, SinUno(conjpos),c)
        FI  
    }
    FI
}

\tadAxioma{entradasCercanas}{ \IF distancia($\pi_2$(pos), alto(g)-1) $<$ distancia($\pi_2$(pos), 0) 
    THEN Ag(($\pi_1$(pos), alto(g)-1) $\emptyset$)
    ELSE {
        \IF distancia($\pi_2$(pos), 0) $<$ distancia($\pi_2$(pos), alto(g)-1)
        THEN Ag(($\pi_1$(pos), 0) $\emptyset$)
        ELSE Ag(($\pi_2$(pos), alto(g)-1), Ag(($\pi_1$(pos), 0) $\emptyset$))
        FI
    }
    FI
}

\tadAxioma{distancia(n,m)}{\IF $m-n < 0$ THEN $-(m-n)$ ELSE $m-n$ FI}

\tadAxioma{estudiantes(crearCampus(g, ass))}{$\emptyset$}
\tadAxioma{estudiantes(entraEstudiante(c, pos))}{
    \IF seConvierteEnHippie?(c, pos) THEN estudiantes(c) - estudiantesCapturados(c, pos)
    ELSE Ag(pos, estudiantes(c) - estudiantesCapturados(c, pos))
    FI
}
\tadAxioma{estudiantes(entraHippie(c, pos))}{estudiantes(c) - estudiantesCapturados(c, pos) - \\estudiantesFalopeados(c, pos)}
\tadAxioma{estudiantes(saleEstudiante(c, pos))}{estudiantes(c) - \{pos\}}
\tadAxioma{estudiantes(movEstudiante(c, ini, fin))}{
    \IF seConvierteEnHippie?(c, fin) THEN estudiantes(c) - \{ini\} - estudiantesCapturados(c, fin)
    ELSE Ag(fin, estudiantes(c) - \{ini\} - estudiantesCapturados(c, fin))
    FI
}
\tadAxioma{estudiantes(movHippie(c, pos))}{
    % saber a donde se mueve el hippie
    % preguntar a quien convierte en hippies en esa nueva posicion y restarlos de estudiantes(c)
    % preguntar a quien acorrala el hippie con estudiantesCapturados de la nueva pos y restarlos tambien
    estudiantes(c) - estudiantesCapturados(c, nuevaPosHippie(c, pos)) - estudiantesFalopeados(c, nuevaPosHippie(c, pos)) 
}
\tadAxioma{estudiantes(movAS(c, id))}{estudiantes(c) - estCapturadosPorAS(c, nuevaPosAS(c, id))}

\tadAxioma{asValidos?(g, ass)}{iterarClaves(g, ass, claves(ass))}
\tadAxioma{iterarClaves(g, ass, claves)}{
    \IF $\emptyset$?(claves) THEN true
    ELSE { 
    \-\- \IF asValido?(g, obtener(ass, dameUno(claves))) $\land$ \\ posNoRepetida?(ass, claves, posicion(dameUno(claves))) THEN iterarClaves(g, ass, sinUno(claves))
    \-\- ELSE false
    \-\- FI
    }
    FI 
}
\tadAxioma{asValido?(g, as)}{
    posValida?(g, posicion(as)) $\land$ posicion(as) $\notin$ obstaculos(g) $\land$ \\
    sinCapturas?(as) $\land$ sinSanciones?(as)
}
\tadAxioma{posValida?(g, pos)}{(x(pos) $<$ ancho(g)) $\land$ (y(pos) $<$ alto(g))}
\tadAxioma{sinCapturas?(as)}{capturas(as) = 0}
\tadAxioma{sinSanciones?(as)}{sanciones(as) = 0}
\tadAxioma{posNoRepetida?(ass, claves, pos)}{
    \IF $\emptyset$?(claves) THEN true
    ELSE posicion(obtener(ass, dameUno(claves))) = pos $\land$ \\
    posNoRepetida?(ass, sinUno(claves), pos)
    FI
}

\tadAxioma{posVacia?(c, pos)}{pos $\notin$ obstaculos(grilla(c)) $\land$ posSinAS?(c, pos) $\land$\\ $\neg$esHippie?(c, pos) $\land$ $\neg$esEstudiante?(c, pos)}

\tadAxioma{movYActualizarAS(p,nuevapos,c)}{
actualizarAs(nuevapos,definir(p,(nuevapos,$\pi_2$(Obtener(p,asshole(c))),$\pi_3$(Obtener(p,asshole(c)))),asshole(c)),c)
}

\tadAxioma{actualizarAs(pos,dic,c)}{
	\IF $\neg$hayHippiesAtrapados(pos,c) $\land \neg$hayEstudiantesAtrapados(pos,c) THEN dic
	ELSE {
		\IF hayHippiesAtrapados(pos,c) $\land \neg$hayEstudiantesAtrapados(pos,c) 
		THEN premiarAs(asParaPremiar(rodeo(posConHippies(posContiguas(pos ,grilla(c)) ,Hippies(c)) ,c) ,c) ,c)
		ELSE{
			\IF	$\neg$hayHippiesAtrapados(pos,c) $\land$ hayEstudiantesAtrapados(pos,c) THEN sancionarAs(pos,claves(dic),c)
			ELSE premiarYSancionarAs(pos,claves(dic),c)
			FI
		}
		FI
	}
	FI
}

\tadAxioma{hayHippiesAtrapados(pos,c)}{
	\IF $\emptyset$?(posConHippies(posContiguas(pos,grilla(c)),Hippies(c))) THEN false
	ELSE revisarHippies(posConHippies(posContiguas(pos,grilla(c)),Hippies(c)),Hippies(c)),c)
	FI
}

\tadAxioma{posConHippies(poscont,conjhippies)}{
	\IF $\emptyset$?(poscont) THEN $\emptyset$
	ELSE{
		\IF DameUno(poscont) $\in$ conjhippies	THEN Ag(DameUno(poscont),posConHippies(SinUno(poscont),conjhippies))
		ELSE posConHippies(SinUno(poscont),conjhippies)
		FI
	}
	FI
}

\tadAxioma{posContiguas(pos,g)}{ posValidasCont(Ag(($\pi_1$(pos)+1,$\pi_2$(pos)), Ag(($\pi_1$(pos)-1,$\pi_2$(pos)), Ag(($\pi_1$(pos),$\pi_2$(pos)+1), Ag(($\pi_1$(pos),$\pi_2$(pos)-1), $\emptyset$),g)
}

\tadAxioma{posValidasCont(conj,g)}{
	\IF $\emptyset$?(conj) THEN $\emptyset$
	ELSE{
		\IF posValida?(g,DameUno(conj)) THEN Ag(DameUno(conj),posValidasCont(g,SinUno(conj)))
		ELSE posValidasCont(g,SinUno(conj))
		FI
	}
	FI
}

\tadAxioma{conjuntoPosAs(conjclaves,dic)}{
	\IF $\emptyset$?(conjclaves) THEN $\emptyset$
	ELSE Ag($\pi_1$(Obtener(DameUno(conjclaves),dic)),conjuntoPosAs(SinUno(conjclaves),dic)
	FI
}

\tadAxioma{revisarHippies(conjpos,c)}{
	\IF $\emptyset$?(conjpos) THEN false
	ELSE{
		\IF esAtrapadoH?(DameUno(conjpos), posContiguas(DameUno(conjpos),grilla(c)),c)
		THEN true
		ELSE revisarHippies(SinUno(conjpos),c)
		FI
	}
	FI
}

\tadAxioma{esAtrapadoH?(posh,conjcont,c)}{
	\IF hayPosVacia?(conjcont,c) THEN false
	ELSE{
		\IF DameUno(conjcont) $\in$ conjuntoPosAs(claves(Asshole(c)),Asshole(c)) THEN true
		ELSE esAtrapadoH?(posh,SinUno(conjcont),c)
		FI	
	}
	FI
}

\tadAxioma{hayPosVacia?(conjpos,c)}{
	\IF $\emptyset$?(conjpos) THEN false
	ELSE{
		\IF estaVaciaPos?(DameUno(conjpos)) THEN true
		ELSE hayPosVacia?(conjpos,c)
		FI	
	}
	FI
}

\tadAxioma{hayEstudiantesAtrapados(pos,c)}{
	\IF $\emptyset$?(posConEstudiantes(posContiguas(pos, grilla(c)), Estudiantes(c))) THEN false
	ELSE revisarEstudiantes(posConEstudiantes(posContiguas(pos, grilla(c)), Estudiantes(c)),c)
	FI
}

\tadAxioma{posConEstudiantes(poscont, conjest)}{
	\IF $\emptyset$?(poscont) THEN $\emptyset$
	ELSE{
		\IF DameUno(poscont) $\in$ conjest THEN Ag(DameUno(poscont), posConEstudiantes(SinUno(poscont), conjest))
		ELSE posConEstudiantes(SinUno(poscont), conjest)
		FI
	}
	FI
}

\tadAxioma{revisarEstudiantes(conjpos,c)}{
	\IF $\emptyset$?(conjpos) THEN false
	ELSE{
		\IF esAtrapadoE?(DameUno(conjpos), posContiguas(DameUno(conjpos),grilla(c)),c))) THEN true
		ELSE revisarEstudiantes(SinUno(conjpos),c)
		FI
	}
	FI
}

\tadAxioma{esAtrapadoE?(posE,conjcont,c)}{
	\IF hayPosVacia?(conjcont,c) $\lor$ estaEnEntrada(posE,grilla(c)) THEN false
	ELSE {
		\IF DameUno(conjcont) $\in$ conjuntoPosAs(claves(asshole(c),asshole(c)) THEN true
		ELSE esAtrapadoE?(posE,SinUno(conjcont),c)
		FI	
	}
	FI
}

\tadAxioma{premiarAs(conjAs,c)}{
	\IF $\emptyset$?(conjAs) THEN asshole(c)
	ELSE definir(DameUno(conjAs), ($\pi_1$(Obtener(DameUno(conjAs),asshole(c)) , $\pi_2$(Obtener(DameUno(conjAs),asshole(c))+1,$\pi_3$(Obtener(DameUno(conjAs),asshole(c)), premiarAs(SinUno(conjAs),c))
	FI
}

\tadAxioma{rodeo(conj,c)}{
	\IF $\emptyset$?(conjAs) THEN $\emptyset$
	ELSE posValidaCont(posContigua(DameUno(conj)),grilla(c)) $\cup$ rodeo(SinUno(conj),c)
	FI
}

\tadAxioma{placaDelAs(pos,conjplacas,dic)}{
	\IF $\pi_1$(Obtener(DameUno(conjplacas),dic) $=$ pos THEN DameUno(conjplacas)
	ELSE placaDelAs(pos,SinUno(conjplacas),dic)
	FI
}

\tadAxioma{asParaPremiar(conjtodo,c)}{
	\IF $\emptyset$?(conjAs) THEN $\emptyset$
	ELSE{
		\IF DameUno(conjtodo) $\in$ conjuntoPosAs(claves(asshole(c),asshole(c))
		THEN Ag(placaDelAs(DameUno(conjtodo),claves(asshole(c),asshole(c)),asParaPremiar(SinUno(conjtodo),c))
		ELSE asParaPremiar(SinUno(conjtodo),c)
		FI
	}
	FI
}

\tadAxioma{hAtrapados(conjH,c)}{
	\IF $\emptyset$?(conjH) THEN $\emptyset$
	ELSE{
		\IF esAtrapadoH(DameUno(conjH), posContiguas(DameUno(conjH),grilla(c)), c)
		THEN Ag(DameUno(conjH), hAtrapados(SinUno(conjH),c)
		ELSE hAtrapados(SinUno(conjH),c)
		FI
	}
	FI
}

\tadAxioma{capturadoH(posH,c)}{ esAtrapadoH?(posH,posContiguas(posH,grilla(c)),c)
}

\tadAxioma{capturadoE(posE,c)}{ esAtrapadoE?(posE,posContiguas(posE,grilla(c)),c)
}

\tadAxioma{estaEnEntrada?(posE,g)}{ $\pi_2$(posE) = 0 $\lor$ $\pi_2$(posE) = altura(g)-1
}

\tadAxioma{distanciaTaxista(pos,posPers)}{ distancia($\pi_1$(pos),$\pi_1$(posPers)) - distancia($\pi_2$(pos),$\pi_2$(posPers))
}

\tadAxioma{menorDistanciaAPos(pos,conj)}{ \IF $\#$(conj) $=1$ THEN distanciaTaxista(pos,DameUno(conj))
	ELSE min(distanciaTaxista(pos,DameUno(conj)),menorDistanciaAPos(pos,conj))
	FI
}

\tadAxioma{personasCercanas(pos,conj)}{ \IF $\emptyset$?(conj) THEN $\emptyset$
	ELSE{
		\IF distanciaTaxista(pos,DameUno(conj))$=$menorDistanciaAPos(pos,conj) THEN Ag(DameUno(conjH),personasCercanas(pos,SinUno(conj)))
		ELSE personasCercanas(pos,SinUno(conj))
		FI
	}
	FI
}

\tadAxioma{sancionarAs(conjAs,c)}{ \IF $\emptyset$?(conjAs) THEN asshole(c)
	ELSE definir(DameUno(conjAs), ($\pi_1$(Obtener(DameUno(conjAs),asshole(c))),$\pi_2$(Obtener(DameUno(conjAs),asshole(c))),$\pi_3$(Obtener(DameUno(conjAs),asshole(c)))$+1$), sancionarAs(SinUno(conjAs),c))
	FI
}

\tadAxioma{eAtrapados(conjE,c)}{ \IF $\emptyset$?(conjAs) THEN $\emptyset$
	ELSE{
		\IF esAtrapadoE?(DameUno(conjE),posContiguas(DameUno(conjE),grilla(c)),c)
		THEN Ag(DameUno(conjE),eAtrapados(SinUno(conjE),c))
		ELSE eAtrapados(SinUno(conjE),c)
		FI
	}
	FI
}

%false-> premia / true -> sanciona
\tadAxioma{premiarYSancionar(conjP,c)}{ \IF $\emptyset$?(conjAs) THEN asshole(c)
	ELSE{
		\IF $\pi_2$(DameUno(conjP))$=$false
		THEN definir(DameUno(conjP), ($\pi_1$(Obtener(DameUno(conjP),asshole(c)) , $\pi_2$(Obtener(DameUno(conjP),asshole(c))+1,$\pi_3$(Obtener(DameUno(conjP),asshole(c)), premiarYSancionar(SinUno(conjP),c))
		ELSE definir(DameUno(conjP), ($\pi_1$(Obtener(DameUno(conjP),asshole(c)) , $\pi_2$(Obtener(DameUno(conjP),asshole(c)),$\pi_3$(Obtener(DameUno(conjP),asshole(c))+1, premiarYSancionar(SinUno(conjP),c))
		FI
	}
	FI
}

\tadAxioma{perseguirHippies(p,c)}{ \IF nuevaPosicion(posAS(c,p), posQueAcercan(c, posAS(c,p), hippiesMasCercanos(posAS(c,p), hippies(c))$=$posAS(c,p) THEN asshole(c)
ELSE movYActualizarAS(p, nuevaPosicion(posAS(c,p), posQueAcercan(c, posAS(c,p), hippiesMasCercanos(posAS(c,p), hippies(c)),c)
FI
}


\end{tad}


\section{TAD \tadNombre{Grilla}}

\begin{tad}{\tadNombre{Grilla}}

\tadGeneros{grilla}

\tadExporta{grilla, generadores, observadores basicos}

\tadIgualdadObservacional{g1}{g2}{grilla}{$(ancho(g1) \igobs ancho(g2))\ \land$\\ $(alto(g1) \igobs alto(g2))\ \land$\\ $(obstaculos(g1) \igobs obstaculos(g2))$}

\tadAlinearFunciones{agregarObstaculo}{grilla, posicion}{}

% Generadores
\tadGeneradores

\tadOperacion{crearGrilla}{nat, nat}{grilla}{}
\tadOperacion{agregarObstaculo}{grilla, posicion}{grilla}{}

% Observadores
\tadObservadores

\tadOperacion{ancho}{grilla}{nat}{}
\tadOperacion{alto}{grilla}{nat}{}
\tadOperacion{obstaculos}{grilla}{conj(posicion)}{}

% Axiomas
\tadAxiomas[\paratodo{nat}{an, al}, \paratodo{grilla}{g}, \paratodo{posicion}{p}]
\tadAlinearAxiomas{obstaculos(agregarObstaculo(g, p))}

\tadAxioma{ancho(crearGrilla(an, al))}{an}
\tadAxioma{ancho(agregarObstaculo(g, p))}{ancho(g)}

\tadAxioma{alto(crearGrilla(an, al))}{al}
\tadAxioma{alto(agregarObstaculo(g, p))}{alto(g)}

\tadAxioma{obstaculos(crearGrilla(an, al))}{$\emptyset$}
\tadAxioma{obstaculos(agregarObstaculo(g, p))}{Ag(p, obstaculos(g))}

\end{tad}

\section{Notas}

Cuando un estudiante entra no puede ser atrapado por un AS ya que tiene libertad de movimiento. Puede salir por la entrada.
\newline estudiantesCapturados devuelve un conunto de los estudiantes que fueron atrapados por concecuencia de que algun estudiante o hippie se haya movido a la posicion pasada como parametro, y que eso causara que quedara encerrado por un AS
\newline estudiantesFalopeados devuelve un conunto de los estudiantes que fueron convertidos al hippismo por concecuencia de que un hippie apareciera en la posicion pasada como parametro
\newline estCapturadosPorAS devuelve el conjunto de estudiantes que un AS atrapa en la posicion pasada como parametro
\end{document}