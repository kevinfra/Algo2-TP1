\documentclass[10pt, a4paper]{article}
\usepackage[paper=a4paper, left=1.5cm, right=1.5cm, bottom=1.5cm, top=3.5cm]{geometry}
\usepackage[latin1]{inputenc}
\usepackage[T1]{fontenc}
\usepackage[spanish]{babel}
\usepackage{indentfirst}
\usepackage{fancyhdr}
\usepackage{latexsym}
\usepackage{lastpage}
\usepackage{aed2-symb,aed2-itef,aed2-tad}
\usepackage[colorlinks=true, linkcolor=blue]{hyperref}
\usepackage{calc}

\newcommand{\f}[1]{\text{#1}}
\renewcommand{\paratodo}[2]{$\forall~#2$: #1}


\sloppy

\hypersetup{%
 % Para que el PDF se abra a página completa.
 pdfstartview= {FitH \hypercalcbp{\paperheight-\topmargin-1in-\headheight}},
 pdfauthor={Grupo MISSINGNO - Algoritmos y Estructuras de Datos II},
 pdfkeywords={Trabajo Practico 1},
 pdfsubject={Alta seguridad nos cuida}
}

\parskip=5pt % 10pt es el tamaño de fuente

% Pongo en 0 la distancia extra entre ítemes.
\let\olditemize\itemize
\def\itemize{\olditemize\itemsep=0pt}

% Acomodo fancyhdr.
\pagestyle{fancy}
\thispagestyle{fancy}
\addtolength{\headheight}{1pt}
\lhead{Algoritmos y Estructuras de Datos II}
\rhead{$1^{\mathrm{er}}$ cuatrimestre de 2012}
\cfoot{\thepage /\pageref{LastPage}}
\renewcommand{\footrulewidth}{0.4pt}

\author{Grupo MISSINGNO - Algoritmos y Estructuras de Datos II}
\date{2do cuatrimestre 2015}
\title{Trabajo Practico 1}

\begin{document}

%Pagina de titulo e indice
\thispagestyle{empty}

\maketitle
\tableofcontents

\newpage

\section{TADs Auxiliares}
\par \textbf{TAD} as \textbf{ES} tupla(posicion $pos$, nat $capturas$, nat $sanciones$)
\par \textbf{TAD} posicion \textbf{ES} tupla(nat $x$, nat $y$)
\par \textbf{TAD} movimiento \textbf{ES} tupla(posicion $pos$, direccion $dir$)
\par \textbf{TAD} direccion \textbf{ES} enum\{arriba, abajo, izquierda, derecha\}
\par \textbf{TAD} placa \textbf{ES} nat

%TADS
\section{TAD \tadNombre{Campus}}

\begin{tad}{\tadNombre{Campus}}

\tadGeneros{campus}

\tadExporta{campus, generadores, observadores basicos}

\tadIgualdadObservacional{c1}{c2}{campus}{$(grilla(c1) \igobs grilla(c2))\ \land$\\ $(asshole(c1) \igobs asshole(c2))\ \land$\\ $(estudiantes(c1) \igobs estudiantes(c2)) \ \land$\\ $(hippies(c1) \igobs hippies(c2))$}

\tadAlinearFunciones{posNoRepetida?}{campus/c, posicion/p, direccion/dir}{}

% Generadores
\tadGeneradores

\tadOperacion{crearCampus}{grilla/g, dicc(placa, as)/ass}{campus}{asValidos?(g, ass)}
\tadOperacion{entraEstudiante}{campus/c, posicion/p}{campus}{posValida?(grilla(c), p) $\land$ posVacia?(c, p) $\land$ esEntrada?(c, p)}
\tadOperacion{entraHippie}{campus/c, posicion/p}{campus}{posValida?(grilla(c), p) $\land$ posVacia?(c, p) $\land$ esEntrada?(c, p)}
\tadOperacion{saleEstudiante}{campus/c, posicion/p}{campus}{esEstudiante?(c, p) $\land$ esEntrada?(c, p)}
\tadOperacion{movEstudiante}{campus/c, posicion/ini, posicion/fin}{campus}{esEstudiante?(c, ini) $\land$ posValida?(c, fin) $\land$ posVacia?(c, fin) $\land$ adyacentes?(ini, fin)}
\tadOperacion{movHippie}{campus/c, posicion/p}{campus}{esHippie?(c, p)}
\tadOperacion{movAS}{campus/c, placa/p}{campus}{esAS?(c, p)}

% Observadores
\tadObservadores

\tadOperacion{grilla}{campus}{grilla}{}
\tadOperacion{asshole}{campus}{dicc(placa, as)}{}
\tadOperacion{estudiantes}{campus}{conj(posicion)}{}
\tadOperacion{hippies}{campus}{conj(posicion)}{}

% Otras operaciones
\tadOtrasOperaciones

\tadOperacion{asValidos?}{grilla, dicc(placa, as)}{bool}{}
\tadOperacion{iterarClaves}{grilla, dicc(placa, as), conj(placa)}{bool}{}
\tadOperacion{posValida?}{grilla, posicion}{bool}{}
\tadOperacion{sinCapturas?}{dicc(placa, as)}{bool}{}
\tadOperacion{sinSanciones?}{dicc(placa, as)}{bool}{}
\tadOperacion{posNoRepetida?}{ass, conj(placa), posicion}{bool}{}

\tadOperacion{posVacia?}{campus, posicion}{bool}{}

\tadAlinearFunciones{movEntradaMasCercana}{posicion, conj(posicion), campus}{}

\tadOperacion{posSinAS?}{campus, posicion}{bool}{}
\tadOperacion{esHippie?}{campus, posicion}{bool}{}
\tadOperacion{esEstudiante?}{campus, posicion}{bool}{}
\tadOperacion{esEntrada?}{campus, posicion}{bool}{}
\tadOperacion{adyacentes?}{posicion, posicion}{bool}{}
\tadOperacion{esAS?}{campus, placa}{bool}{}
\tadOperacion{nuevaPosAS}{campus, placa}{posicion}{}
\tadOperacion{hayHippies?}{campus}{bool}{}
\tadOperacion{movEntradaMasCercana}{campus, placa}{dicc(placa,as)}{}
\tadOperacion{posAS}{campus, placa}{posicion}{}
\tadOperacion{nuevaPosicion}{posicion, conj(posicion),campus}{posicion}{}
\tadOperacion{entradasMasCercanas}{posicion,grilla}{conj(posicion)}{}
\tadOperacion{distancia}{nat,nat}{nat}{}
\tadOperacion{seConvierteEnHippie?}{campus, posicion}{bool}{}
\tadOperacion{estudiantesCapturados}{campus, posicion}{conj(posicion)}{}
\tadOperacion{estudiantesFalopeados}{campus, posicion}{conj(posicion)}{}
\tadOperacion{nuevaPosHippie}{campus, posicion}{posicion}{}

\tadOperacion{cantidadDeHippies}{campus}{nat}{}
\tadOperacion{cantidadDeEstudiantes}{campus}{nat}{}
\tadOperacion{asMasVigilante}{campus}{placa}{}
\tadOperacion{conjuntoDeCapturas}{dicc(placa,as)}{conj(nat)}{}
\tadOperacion{DameMasVigilantes}{conj(nat)}{nat}{}
\tadOperacion{PlacasPosibles}{dicc(placa,as)/ass, nat/NDeCapt}{conj(placa)}{nDeCaptValido(NDeCapt,ass)} %hay que hacer la restriccion
\tadOperacion{elDeMenorPlaca}{conj(placa)}{placa}{}

% Axiomas
\tadAxiomas[\paratodo{campus}{c}, \paratodo{grilla}{g}, \paratodo{posicion}{p}, \paratodo{direccion}{dir}, \paratodo{dicc(placa,as)}{ass}, \paratodo{placa}{id}]
\tadAlinearAxiomas{estudiantes(movEstudiante(c, ini, fin))}

\tadAxioma{grilla(crearCampus(g, ass))}{g}
\tadAxioma{grilla(entraEstudiante(c, p))}{grilla(c)}
\tadAxioma{grilla(entraHippie(c, p))}{grilla(c)}
\tadAxioma{grilla(saleEstudiante(c, p))}{grilla(c)}
\tadAxioma{grilla(movEstudiante(c, p, dir))}{grilla(c)}
\tadAxioma{grilla(movHippie(c, p))}{grilla(c)}
\tadAxioma{grilla(movAS(c, id))}{grilla(c)}

\tadAxioma{asshole(crearCampus(g, ass))}{ass}
\tadAxioma{asshole(entraEstudiante(c, p))}{asshole(c)}
\tadAxioma{asshole(entraHippie(c, p))}{asshole(c)}
\tadAxioma{asshole(saleEstudiante(c, p))}{asshole(c)}
\tadAxioma{asshole(movEstudiante(c, p, dir))}{asshole(c)}
\tadAxioma{asshole(movHippie(c, p))}{asshole(c)}
\tadAxioma{asshole(movAS(c, p))}{ nuevaPosAs(c,p)
}

\tadAxioma{posAS(c, p)}{$pi_1$(Obtener(p,asshole(c)))}

\tadAxioma{hayHippies?(c)}{\IF $\#$Hippies(c) $> 0$ THEN true ELSE false FI}

\tadAxioma{movEntradaMasCercana(c, p)}{\IF nuevaPosicion(posAS(c, p), entradasMasCercanas(posAS(c, p),grilla(c)))= posAS(c, p)
    THEN asshole(c)
    ELSE moverseYActualizar(p,nuevaPosicion(posAS(c, p), entradasMasCercanas(posAS(c, p),grilla(c))),c)
    FI
}

\tadAxioma{nuevaPosicion(posactual,conjpos,c)}{\IF $\emptyset$?(conjpos) THEN posactual
    ELSE {
    \IF estaVaciaPos?(DameUno(conjpos),c) THEN DameUno(conjpos)
        ELSE nuevaPosicion(posactual, SinUno(conjpos),c)
        FI  
    }
    FI
}

\tadAxioma{entradasMasCercanas}{ \IF distancia($\pi_2$(pos), alto(g)-1) $<$ distancia($\pi_2$(pos), 0) 
    THEN Ag(($\pi_1$(pos), alto(g)-1) $\emptyset$)
    ELSE {
        \IF distancia($\pi_2$(pos), 0) $<$ distancia($\pi_2$(pos), alto(g)-1)
        THEN Ag(($\pi_1$(pos), 0) $\emptyset$)
        ELSE Ag(($\pi_2$(pos), alto(g)-1), Ag(($\pi_1$(pos), 0) $\emptyset$))
        FI
    }
    FI
}

\tadAxioma{distancia(n,m)}{\IF $m-n < 0$ THEN $-(m-n)$ ELSE $m-n$ FI}

\tadAxioma{estudiantes(crearCampus(g, ass))}{$\emptyset$}
\tadAxioma{estudiantes(entraEstudiante(c, pos))}{
    \IF seConvierteEnHippie?(c, pos) THEN estudiantes(c) - estudiantesCapturados(c, pos)
    ELSE Ag(pos, estudiantes(c) - estudiantesCapturados(c, pos))
    FI
}
\tadAxioma{estudiantes(entraHippie(c, pos))}{estudiantes(c) - estudiantesCapturados(c, pos) - \\estudiantesFalopeados(c, pos)}
\tadAxioma{estudiantes(saleEstudiante(c, pos))}{estudiantes(c) - \{pos\}}
\tadAxioma{estudiantes(movEstudiante(c, ini, fin))}{
    \IF seConvierteEnHippie?(c, fin) THEN estudiantes(c) - \{ini\} - estudiantesCapturados(c, fin)
    ELSE Ag(fin, estudiantes(c) - \{ini\} - estudiantesCapturados(c, fin))
    FI
}
\tadAxioma{estudiantes(movHippie(c, pos))}{
    % saber a donde se mueve el hippie
    % preguntar a quien convierte en hippies en esa nueva posicion y restarlos de estudiantes(c)
    % preguntar a quien acorrala el hippie con estudiantesCapturados de la nueva pos y restarlos tambien
    estudiantes(c) - estudiantesCapturados(c, nuevaPosHippie(c, pos)) - estudiantesFalopeados(c, nuevaPosHippie(c, pos)) 
}
\tadAxioma{estudiantes(movAS(c, id))}{estudiantes(c) - estCapturadosPorAS(c, nuevaPosAS(c, id))}

\tadAxioma{asValidos?(g, ass)}{iterarClaves(g, ass, claves(ass))}
\tadAxioma{iterarClaves(g, ass, claves)}{
    \IF $\emptyset$?(claves) THEN true
    ELSE { 
    \-\- \IF asValido?(g, obtener(ass, dameUno(claves))) $\land$ \\ posNoRepetida?(ass, claves, posicion(dameUno(claves))) THEN iterarClaves(g, ass, sinUno(claves))
    \-\- ELSE false
    \-\- FI
    }
    FI 
}
\tadAxioma{asValido?(g, as)}{
    posValida?(g, posicion(as)) $\land$ posicion(as) $\notin$ obstaculos(g) $\land$ \\
    sinCapturas?(as) $\land$ sinSanciones?(as)
}
\tadAxioma{posValida?(g, pos)}{(x(pos) $<$ ancho(g)) $\land$ (y(pos) $<$ alto(g))}
\tadAxioma{sinCapturas?(as)}{capturas(as) = 0}
\tadAxioma{sinSanciones?(as)}{sanciones(as) = 0}
\tadAxioma{posNoRepetida?(ass, claves, pos)}{
    \IF $\emptyset$?(claves) THEN true
    ELSE posicion(obtener(ass, dameUno(claves))) = pos $\land$ \\
    posNoRepetida?(ass, sinUno(claves), pos)
    FI
}

\tadAxioma{posVacia?(c, pos)}{pos $\notin$ obstaculos(grilla(c)) $\land$ posSinAS?(c, pos) $\land$\\ $\neg$esHippie?(c, pos) $\land$ $\neg$esEstudiante?(c, pos)}

\tadAxioma{nuevaPosAS(c, p)}{ \IF sanciones((Obtener(p,asshole(c)))) $\geq3$ 
    THEN asshole(c)
    ELSE {
        \- \IF $\neg$hayHippies?(c) THEN movEntradaMasCercana(c, p)
        ELSE perseguirHippiesCercanos(p,c)
        FI
    }
    FI
}

\tadAxioma{cantidadDeHippies(c)}{ $\#$ hippies(c)}
\tadAxioma{cantidadDeEstudiantes(c)}{ $\#$ estudiantes(c)}
\tadAxioma{asMasVigilante(c)}{elDeMenorPlaca(PlacasPosibles(asshole(c), DameMasVigilantes(conjuntoDeCapturas(asshole(c)))))}

\tadAxioma{conjuntoDeCapturas(asshole}{\IF $\#$ claves(asshole) $= 1$ THEN capturas(obtener(DameUno(claves(asshole)),asshole)) ELSE Ag(capturas(obtener(DameUno(claves(asshole)))), conjuntoDeCapturas(borrar(DameUno(claves(asshole))), asshole)) FI}

\tadAxioma{DameMasVigilantes(conjCapturas)}{\IF $\#$conjCapturas $= 1$ THEN DameUno(conjCapturas) 
ELSE max(DameUno(conjCapturas), DameMasVigilantes(SinUno(conjCapturas))) FI }

\tadAxioma{PlacasPosibles(asshole, numeroDeCapturas)}{ \IF $\#$claves(asshole) $= 1$ 
	THEN DameUno(claves(asshole)) 
	ELSE{
	 \- \IF capturas(DameUno(claves(asshole))) $=$ numeroDeCapturas THEN Ag(DameUno(claves(asshole)), PlacasPosibles(borrar(DameUno(claves(asshole)), asshole), numeroDeCapturas))
	 	 ELSE PlacasPosibles(borrar(DameUno(claves(asshole)), asshole), numeroDeCapturas) 
	 	 FI
	}
	FI
} %hay que hacer la restriccion que diga que numeroDeCapturas es un numero perteneciente al conjunto de capturas de todas las claves del Dicc 

\tadAxioma{elDeMenorPlaca(conjPlacas)}{\IF $\#$conjPlacas $= 1$ 
	THEN DameUno(conjPlacas) 
	ELSE min(DameUno(conjPlacas), elDeMenorPlaca(SinUno(conjPlacas))) 
	FI 
}


\end{tad}


\section{TAD \tadNombre{Grilla}}

\begin{tad}{\tadNombre{Grilla}}

\tadGeneros{grilla}

\tadExporta{grilla, generadores, observadores basicos}

\tadIgualdadObservacional{g1}{g2}{grilla}{$(ancho(g1) \igobs ancho(g2))\ \land$\\ $(alto(g1) \igobs alto(g2))\ \land$\\ $(obstaculos(g1) \igobs obstaculos(g2))$}

\tadAlinearFunciones{agregarObstaculo}{grilla, posicion}{}

% Generadores
\tadGeneradores

\tadOperacion{crearGrilla}{nat, nat}{grilla}{}
\tadOperacion{agregarObstaculo}{grilla, posicion}{grilla}{}

% Observadores
\tadObservadores

\tadOperacion{ancho}{grilla}{nat}{}
\tadOperacion{alto}{grilla}{nat}{}
\tadOperacion{obstaculos}{grilla}{conj(posicion)}{}

% Axiomas
\tadAxiomas[\paratodo{nat}{an, al}, \paratodo{grilla}{g}, \paratodo{posicion}{p}]
\tadAlinearAxiomas{obstaculos(agregarObstaculo(g, p))}

\tadAxioma{ancho(crearGrilla(an, al))}{an}
\tadAxioma{ancho(agregarObstaculo(g, p))}{ancho(g)}

\tadAxioma{alto(crearGrilla(an, al))}{al}
\tadAxioma{alto(agregarObstaculo(g, p))}{alto(g)}

\tadAxioma{obstaculos(crearGrilla(an, al))}{$\emptyset$}
\tadAxioma{obstaculos(agregarObstaculo(g, p))}{Ag(p, obstaculos(g))}

\end{tad}

\section{Notas}

Cuando un estudiante entra no puede ser atrapado por un AS ya que tiene libertad de movimiento. Puede salir por la entrada.
\newline estudiantesCapturados devuelve un conunto de los estudiantes que fueron atrapados por concecuencia de que algun estudiante o hippie se haya movido a la posicion pasada como parametro, y que eso causara que quedara encerrado por un AS
\newline estudiantesFalopeados devuelve un conunto de los estudiantes que fueron convertidos al hippismo por concecuencia de que un hippie apareciera en la posicion pasada como parametro
\newline estCapturadosPorAS devuelve el conjunto de estudiantes que un AS atrapa en la posicion pasada como parametro
\end{document}