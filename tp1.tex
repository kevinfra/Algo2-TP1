\documentclass[10pt, a4paper]{article}
\usepackage[paper=a4paper, left=1.5cm, right=1.5cm, bottom=1.5cm, top=3.5cm]{geometry}
\usepackage[latin1]{inputenc}
\usepackage[T1]{fontenc}
\usepackage[spanish]{babel}
\usepackage{indentfirst}
\usepackage{fancyhdr}
\usepackage{latexsym}
\usepackage{lastpage}
\usepackage{aed2-symb,aed2-itef,aed2-tad}
\usepackage[colorlinks=true, linkcolor=blue]{hyperref}
\usepackage{calc}
\usepackage{caratula/caratula/caratula}

\newcommand{\f}[1]{\text{#1}}
\renewcommand{\paratodo}[2]{$\forall~#2$: #1}


\sloppy

\hypersetup{%
 % Para que el PDF se abra a página completa.
 pdfstartview= {FitH \hypercalcbp{\paperheight-\topmargin-1in-\headheight}},
 pdfauthor={Grupo 1 - Algoritmos y Estructuras de Datos II},
 pdfkeywords={Trabajo Practico 1},
 pdfsubject={Alta seguridad nos cuida}
}

\parskip=5pt % 10pt es el tamaño de fuente

% Pongo en 0 la distancia extra entre ítemes.
\let\olditemize\itemize
\def\itemize{\olditemize\itemsep=0pt}

% Acomodo fancyhdr.
\pagestyle{fancy}
\thispagestyle{fancy}
\addtolength{\headheight}{1pt}
\lhead{Algoritmos y Estructuras de Datos II}
\rhead{$2^{\mathrm{do}}$ cuatrimestre de 2015}
\cfoot{\thepage /\pageref{LastPage}}
\renewcommand{\footrulewidth}{0.4pt}

\author{Grupo MISSINGNO - Algoritmos y Estructuras de Datos II}
\date{2do cuatrimestre 2015}
\title{Trabajo Practico 1}


\begin{document}

\titulo{Trabajo Pr\'actico 1}
\fecha{06 de Septiembre de 2015}
\materia{Algoritmos y Estructuras de Datos II}
\subtitulo{Grupo N\'umero 1}

\integrante{Joel Esteban Camera}{257/14}{joel.e.camera@gmail.com}
\integrante{Manuel Mena}{313/14}{manuelmena1993@gmail.com}
\integrante{Kevin Frachtenberg Goldsmit}{247/14}{kevinfra94@gmail.com}
\integrante{Nicol\'as Bukovits}{546/14}{nicobuk@gmail.com}

%Carátula

%Pagina de titulo e indice
\thispagestyle{empty}

\maketitle
\tableofcontents

\newpage

\section{TADs Auxiliares}
\par \textbf{TAD} as \textbf{ES} tupla(posicion $pos$, nat $capturas$, nat $sanciones$)
\par \textbf{TAD} posicion \textbf{ES} tupla(nat $x$, nat $y$)
\par \textbf{TAD} movimiento \textbf{ES} tupla(posicion $pos$, direccion $dir$)
\par \textbf{TAD} direccion \textbf{ES} enum\{arriba, abajo, izquierda, derecha\}
\par \textbf{TAD} placa \textbf{ES} nat

%TADS
\section{TAD \tadNombre{Campus}}

\begin{tad}{\tadNombre{Campus}}

\tadGeneros{campus}

\tadExporta{campus, generadores, observadores basicos}

\tadIgualdadObservacional{c1}{c2}{campus}{$(grilla(c1) \igobs grilla(c2))\ \land$\\ $(agentes(c1) \igobs agentes(c2))\ \land$\\ $(estudiantes(c1) \igobs estudiantes(c2)) \ \land$\\ $(hippies(c1) \igobs hippies(c2))$}

\tadAlinearFunciones{posNoRepetida?}{campus/c, posicion/p, direccion/dir}{}

% Generadores
\tadGeneradores

\tadOperacion{crearCampus}{grilla/g, dicc(placa, as)/ass}{campus}{asValidos?(g, ass)}
\tadOperacion{entraEstudiante}{campus/c, posicion/p}{campus}{posValida?(grilla(c), p) $\land$ posVacia?(c, p) $\land$ esEntrada?(c, p)}
\tadOperacion{entraHippie}{campus/c, posicion/p}{campus}{posValida?(grilla(c), p) $\land$ posVacia?(c, p) $\land$ esEntrada?(c, p)}
\tadOperacion{saleEstudiante}{campus/c, posicion/p}{campus}{esEstudiante?(c, p) $\land$ esEntrada?(c, p)}
\tadOperacion{movEstudiante}{campus/c, posicion/ini, posicion/fin}{campus}{esEstudiante?(c, ini) $\land$ posValida?(c, fin) $\land$ posVacia?(c, fin) $\land$ adyacentes?(ini, fin)}
\tadOperacion{movHippie}{campus/c, posicion/p}{campus}{esHippie?(c, p)}
\tadOperacion{movAS}{campus/c, placa/p}{campus}{esAS?(c, p)}

% Observadores
\tadObservadores

\tadOperacion{grilla}{campus}{grilla}{}
\tadOperacion{agentes}{campus}{dicc(placa, as)}{}
\tadOperacion{estudiantes}{campus}{conj(posicion)}{}
\tadOperacion{hippies}{campus}{conj(posicion)}{}

% Otras operaciones
\tadOtrasOperaciones

\tadOperacion{asValidos?}{grilla, dicc(placa, as)}{bool}{}
\tadOperacion{iterarClaves}{grilla, dicc(placa, as), conj(placa)}{bool}{}
\tadOperacion{posValida?}{grilla, posicion}{bool}{}
\tadOperacion{sinCapturas?}{dicc(placa, as)}{bool}{}
\tadOperacion{sinSanciones?}{dicc(placa, as)}{bool}{}
\tadOperacion{posNoRepetida?}{ass, conj(placa), posicion}{bool}{}
\tadOperacion{posVacia?}{campus, posicion}{bool}{}

\tadAlinearFunciones{hayEstudiantesAtrapados?}{posicion,conj(placa),dicc(placa,as)}{}

\tadOperacion{posSinAS?}{campus, posicion}{bool}{}
\tadOperacion{conjuntoPosAs}{conj(placa),dicc(placa,as)}{conj(posicion)}{}
\tadOperacion{esHippie?}{campus, posicion}{bool}{}
\tadOperacion{esEstudiante?}{campus, posicion}{bool}{}
\tadOperacion{esEntrada?}{campus, posicion}{bool}{}
\tadOperacion{adyacentes?}{posicion, posicion}{bool}{}
\tadOperacion{esAS?}{campus, placa}{bool}{}

\tadOperacion{actualizarAs}{posicion,dicc(placa,as),campus}{dicc(placa,as)}{}

\tadOperacion{hayHippiesAtrapados}{posicion,campus}{bool}{}
\tadOperacion{posContiguas}{posicion,grilla}{conj(posicion)}{}
\tadOperacion{posValidasCont}{conj(posicion),grilla}{conj(posicion)}{}

\tadOperacion{hayEstudiantesAtrapados?}{posicion,campus}{bool}{}
\tadOperacion{eAtrapados}{conj(posicion),campus}{conj(posicion)}{}
\tadOperacion{esAtrapadoE?}{posicion,conj(posicion),campus}{bool}{}
\tadOperacion{hayPosVacia?}{conj(posicion), campus}{bool}{}
\tadOperacion{estaEnEntrada?}{posicion,grilla}{bool}{}

\tadOperacion{premiarAs}{multiconj(placa),campus}{dicc(placa,as)}{}
\tadOperacion{asParaModificar}{multiconj(posicion),campus}{multiconj(placa)}{}
\tadOperacion{placaDelAs}{posicion,conj(placa),dicc(placa,as)}{placa}{}
\tadOperacion{periferia}{conj(posicion),campus}{multiconj(posicion)}{}
\tadOperacion{hAtrapados}{conj(posicion),campus}{conj(posicion)}{}
\tadOperacion{esAtrapadoH?}{conj(posicion), campus}{bool}{}

\tadOperacion{sancionarAs}{multiconj(placa),campus}{dicc(placa,as)}{}
\tadOperacion{premiarYSancionar}{multiconj((placa,bool)),campus}{dicc(placa,as)}{}
\tadOperacion{asSancPrem}{multiconj(posicion),multiconj(posicion)}{multiconj((placa,bool))}{}
\tadOperacion{hayHippies?}{campus}{bool}{}

\tadOperacion{movEntradaCercana}{campus, placa}{dicc(placa,as)}{}

\tadOperacion{nuevaPosicion}{posicion, conj(posicion),campus}{posicion}{}
\tadOperacion{posAS}{campus, placa}{posicion}{}
\tadOperacion{posQueAcercan}{campus, posicion, conj(posicion)}{conj(posicion)}{}
\tadOperacion{posQueAcercanAUno}{campus, posicion, posicion}{conj(posicion)}{}
\tadOperacion{iterarDirecciones}{posicion, conj(posicion)}{conj(posicion)}{}

\tadOperacion{distTaxi}{posicion,posicion}{nat}{}
\tadOperacion{distancia}{nat,nat}{nat}{}

\tadOperacion{entradasCercanas}{posicion,grilla}{conj(posicion)}{}

\tadOperacion{movYActualizarAS}{placa,posicion,campus}{dicc(placa,as)}{}

\tadOperacion{perseguirHippies}{placa,campus}{dic(placa,as)}{}
\tadOperacion{personasCercanas}{posicion,conj(posicion)}{conj(posicion)}{}
\tadOperacion{menorDistanciaAPos}{posicion,con(posicion)}{nat}{}

\tadOperacion{seConvierteEnHippie?}{campus, posicion}{bool}{}
\tadOperacion{cantHippiesAdyacentes}{campus, posicion}{bool}{}
\tadOperacion{contarHippies}{campus, conj(posicion)}{nat}{}

\tadOperacion{estudiantesFalopeados}{campus, posicion}{conj(posicion)}{}
\tadOperacion{buscarFalopeados}{campus, conj(posicion)}{conj(posicion)}{}

\tadOperacion{nuevaPosHippie}{campus, posicion}{posicion}{}

\tadOperacion{hippiesConvertidos}{campus, posicion}{conj(posicion)}{}
\tadOperacion{buscarConvertidos}{campus, conj(posicion)}{conj(posicion)}{}
\tadOperacion{seConvierteEnEstu?}{campus, posicion}{bool}{}
\tadOperacion{cantEstuAdyacentes}{campus, posicion}{nat}{}
\tadOperacion{contarEstudiantes}{campus, conj(posicion)}{nat}{}

\tadOperacion{hippiesCapturados}{campus, posicion}{conj(posicion)}{}
\tadOperacion{buscarCapturados}{campus, conj(posicion)}{conj(posicion)}{}
\tadOperacion{capturadoH}{posicion,campus}{bool}{}

\tadOperacion{cantidadDeHippies}{campus}{nat}{}

\tadOperacion{cantidadDeEstudiantes}{campus}{nat}{}

\tadOperacion{asMasVigilante}{campus}{placa}{}
\tadOperacion{conjuntoDeCapturas}{dicc(placa,as)}{conj(nat)}{}
\tadOperacion{numCapturasMax}{conj(nat)}{nat}{}
\tadOperacion{losMasVigilantes}{dicc(placa,as), nat}{conj(placa)}{}
\tadOperacion{elDeMenorPlaca}{conj(placa)}{placa}{}

\tadOperacion{nuevaPosAS}{campus, placa}{posicion}{}


% Axiomas
\tadAxiomas[\paratodo{campus}{c}, \paratodo{grilla}{g}, \paratodo{posicion}{p}, \paratodo{direccion}{dir}, \paratodo{dicc(placa,as)}{ass}, \paratodo{placa}{id}]

\tadAlinearAxiomas{agentes(movEstudiante(c, posE, dir))}

\tadAxioma{asValidos?(g, ass)}{iterarClaves(g, ass, claves(ass))}
\tadAxioma{iterarClaves(g, ass, claves)}{
    \IF $\emptyset$?(claves) THEN true
    ELSE { 
    \-\- \IF asValido?(g, obtener(ass, dameUno(claves))) $\land$ \\ posNoRepetida?(ass, claves, posicion(dameUno(claves))) THEN iterarClaves(g, ass, sinUno(claves))
    \-\- ELSE false
    \-\- FI
    }
    FI 
}
\tadAxioma{asValido?(g, as)}{posValida?(g, posicion(as)) $\land$ posicion(as) $\notin$ obstaculos(g) $\land$ \\sinCapturas?(as) $\land$ sinSanciones?(as)}
\tadAxioma{posValida?(g, pos)}{(x(pos) $<$ ancho(g)) $\land$ (y(pos) $<$ alto(g))}
\tadAxioma{sinCapturas?(as)}{capturas(as) = 0}
\tadAxioma{sinSanciones?(as)}{sanciones(as) = 0}
\tadAxioma{posNoRepetida?(ass, claves, pos)}{
    \IF $\emptyset$?(claves) THEN true
    ELSE posicion(obtener(ass, dameUno(claves))) = pos $\land$ \\posNoRepetida?(ass, sinUno(claves), pos)
    FI
}
\tadAxioma{posVacia?(c, pos)}{pos $\notin$ obstaculos(grilla(c)) $\land$ posSinAS?(c, pos) $\land$\\ $\neg$esHippie?(c, pos) $\land$ $\neg$esEstudiante?(c, pos)}
\tadAxioma{posSinAS?(c, pos)}{ pos $\notin$ conjuntoPosAs(claves(agentes(c),agentes(c)))}
\tadAxioma{conjuntoPosAs(conjclaves,dic)}{
    \IF $\emptyset$?(conjclaves) THEN $\emptyset$
    ELSE Ag(Obtener(dameUno(conjclaves),dic).posicion,\\ conjuntoPosAs(sinUno(conjclaves),dic)
    FI
}
\tadAxioma{esHippie?(c, pos)}{ pos $\in$ hippies(c)}
\tadAxioma{esEstudiante?(c, pos)}{ pos $\in$ estudiantes(c)}
\tadAxioma{esEntrada?(c,pos)}{ pos.y = 0 $\lor$ pos.y = alto(grilla(c)) - 1 }
\tadAxioma{adyacentes?(pos1, pos2)}{ 
    (pos1.y = pos2.y + 1 $\land$ pos1.x = pos2.x) $\lor$ \\
    (pos2.y = pos1.y + 1 $\land$ pos1.x = pos2.x) $\lor$ \\    
    (pos1.x = pos2.x + 1 $\land$ pos1.y = pos2.y) $\lor$ \\    
    (pos2.x = pos1.x + 1 $\land$ pos1.y = pos2.y)        
}
\tadAxioma{esAS?(c, p)}{ def?(p,agentes(c))}

\tadAxioma{grilla(crearCampus(g, ass))}{g}
\tadAxioma{grilla(entraEstudiante(c, p))}{grilla(c)}
\tadAxioma{grilla(entraHippie(c, p))}{grilla(c)}
\tadAxioma{grilla(saleEstudiante(c, p))}{grilla(c)}
\tadAxioma{grilla(movEstudiante(c, p, dir))}{grilla(c)}
\tadAxioma{grilla(movHippie(c, p))}{grilla(c)}
\tadAxioma{grilla(movAS(c, id))}{grilla(c)}

\tadAxioma{agentes(crearCampus(g, ass))}{ass}
\tadAxioma{agentes(entraEstudiante(c, posE))}{actualizarAs(posE,agentes(c),c)}
\tadAxioma{agentes(entraHippie(c, posH))}{actualizarAs(posH,agentes(c),c)}
\tadAxioma{agentes(saleEstudiante(c, posE))}{agentes(c)}
\tadAxioma{agentes(movEstudiante(c, posE, dir))}{actualizarAs(posE,agentes(c),c)}
\tadAxioma{agentes(movHippie(c, posH))}{actualizarAs(posH,agentes(c),c)}

\tadAlinearAxiomas{estudiantes(movEstudiante(c, ini, fin))}

\tadAxioma{agentes(movAS(c, p))}{ \IF obtener(p,agentes(c)).sanciones $\geq3$ 
    THEN agentes(c)
    ELSE {
        \- \IF $\neg$hayHippies?(c) THEN movEntradaCercana(c, p)
        ELSE perseguirHippies(p,c)
        FI
    }
    FI
}

\tadAxioma{estudiantes(crearCampus(g, ass))}{$\emptyset$}
\tadAxioma{estudiantes(entraEstudiante(c, pos))}{
    \IF seConvierteEnHippie?(c, pos) THEN estudiantes(c)
    ELSE Ag(pos, estudiantes(c)
    FI
}
\tadAxioma{estudiantes(entraHippie(c, pos))}{estudiantes(c) - estudiantesFalopeados(c, pos)}
\tadAxioma{estudiantes(saleEstudiante(c, pos))}{estudiantes(c) - \{pos\}}
\tadAxioma{estudiantes(movEstudiante(c, ini, fin))}{
    \IF seConvierteEnHippie?(c, fin) THEN estudiantes(c) - \{ini\} 
    ELSE Ag(fin, estudiantes(c) - \{ini\}) 
    FI
}
\tadAxioma{estudiantes(movHippie(c, pos))}{
    % saber a donde se mueve el hippie
    % preguntar a quien convierte en hippies en esa nueva posicion y restarlos de estudiantes(c)
    % preguntar a quien acorrala el hippie ceva pos y restarlos tambien
    estudiantes(c) - estudiantesFalopeados(c, nuevaPosHippie(c, pos)) 
}
\tadAxioma{estudiantes(movAS(c, id))}{estudiantes(c)}

\tadAxioma{hippies(crearCampus(g, ass))}{$\emptyset$}
\tadAxioma{hippies(entraEstudiante(c, pos))}{hippies(c) - hippiesConvertidos(c, pos) - hippiesCapturados(c, pos)}
\tadAxioma{hippies(entraHippie(c, pos))}{
    \IF capturadoH?(c, pos) THEN hippies(c) - hippiesCapturados(c, pos)
    ELSE Ag(pos, hippies(c) - hippiesCapturados(c, pos)) 
    FI
}
\tadAxioma{hippies(saleEstudiante(c, pos))}{hippies(c)}
\tadAxioma{hippies(movEstudiante(c, ini, fin))}{
    \IF seConvierteEnHippie?(c, fin) THEN Ag(fin, hippies(c) - hippiesCapturados(c, fin))
    ELSE hippies(c) - hippiesCapturados(c, fin)
    FI
}
\tadAxioma{hippies(movHippie(c, pos))}{hippies(c) - hippiesCapturados(c, nuevaPosHippie(c, pos))}
\tadAxioma{hippies(movAS(c, id))}{hippies(c) - hipiesCapturados(c, nuevaPosAS(c, id))}

\tadAxioma{actualizarAs(pos,dic,c)}{
    \IF $\neg$hayHippiesAtrapados(pos,c) $\land \neg$hayEstudiantesAtrapados?(pos,c) THEN dic
    ELSE {
        \IF hayHippiesAtrapados(pos,c) $\land \neg$hayEstudiantesAtrapados?(pos,c) 
        THEN premiarAs(asParaModificar(periferia(hAtrapados(\\ posContiguas(pos ,grilla(c)), c), c), c), c)
        ELSE{
            \IF $\neg$hayHippiesAtrapados(pos,c)$\land$hayEstudiantesAtrapados?(pos,c) THEN sancionarAs(asParaModificar(periferia\\ (eAtrapados(posContiguas(pos ,grilla(c)), c), c), c), c)
            ELSE premiarYSancionarAs(asSancPrem(\\ asParaModificar(periferia(hAtrapados\\ (posContiguas(pos,grilla(c)),c),c),c),\\ asParaModificar(periferia(eAtrapados\\ (posContiguas(pos,grilla(c)),c),c),c)), c)
            FI
        }
        FI
    }
    FI
}
\tadAxioma{hayHippiesAtrapados(pos,c)}{ $\emptyset$?(hAtrapados(posContiguas(pos, grilla(c)), c))}
\tadAxioma{posContiguas(pos,g)}{ posValidasCont(Ag((pos.x+1,pos.y), Ag((pos.x-1,pos.y), Ag((pos.x,pos.y+1), Ag((pos.x,pos.y 1), $\emptyset$),g)}
\tadAxioma{posValidasCont(conj,g)}{
    \IF $\emptyset$?(conj) THEN $\emptyset$
    ELSE{
        \IF posValida?(g,dameUno(conj)) THEN Ag(dameUno(conj),posValidasCont(g,sinUno(conj)))
        ELSE posValidasCont(g,sinUno(conj))
        FI
    }
    FI
}

\tadAxioma{hayEstudiantesAtrapados?(pos,c)}{ $\emptyset$?(eAtrapados(posContiguas(pos, grilla(c)), c))}
\tadAxioma{eAtrapados(conjContiguo,c)}{ 
    \IF $\emptyset$?(conjContiguo) THEN $\emptyset$
    ELSE{
        \IF dameUno(conjContiguo) $\in$ estudiantes(c)
        THEN{
            \IF esAtrapadoE?(dameUno(conjContiguo),\\ posContiguas(dameUno(conjContiguo),grilla(c)))
            THEN Ag(dameUno(conjContiguo),\\ eAtrapados(sinUno(conContiguo),c))
            ELSE eAtrapados(sinUno(conjContiguo),c)
            FI
        }
        ELSE eAtrapados(sinUno(conjContiguo),c)
        FI
    }
    FI
}
\tadAxioma{esAtrapadoE?(posE,conjcont,c)}{
    \IF $\emptyset$?(conjcont) THEN false
    ELSE{
        \IF hayPosVacia?(conjcont,c) $\lor$ estaEnEntrada(posE,grilla(c)) THEN false
        ELSE {
            \IF dameUno(conjcont) $\in$\\ conjuntoPosAs(claves(agentes(c),agentes(c)) THEN true
            ELSE esAtrapadoE?(posE,sinUno(conjcont),c)
            FI  
        }
        FI
    }
    FI
}
\tadAxioma{hayPosVacia?(conjpos,c)}{
    \IF $\emptyset$?(conjpos) THEN false
    ELSE{
        \IF posVacia?(dameUno(conjpos)) THEN true
        ELSE hayPosVacia?(conjpos,c)
        FI  
    }
    FI
}
\tadAxioma{estaEnEntrada?(posE,g)}{ posE.y = 0 $\lor$ posE.y = altura(g) - 1}

\tadAxioma{premiarAs(conjAs,c)}{
    \IF $\emptyset$?(conjAs) THEN agentes(c)
    ELSE definir(dameUno(conjAs),\\ (Obtener(dameUno(conjAs),agentes(c)).posicion,\\ Obtener(dameUno(conjAs),agentes(c)).capturas +1,\\ Obtener(dameUno(conjAs),agentes(c)).sanciones),\\ premiarAs(sinUno(conjAs),c))
    FI
}
\tadAxioma{asParaModificar(conjtodo,c)}{
    \IF $\emptyset$?(conjAs) THEN $\emptyset$
    ELSE{
        \IF dameUno(conjtodo) $\in$\\ conjuntoPosAs(claves(agentes(c),agentes(c))
        THEN Ag(placaDelAs(dameUno(conjtodo),claves(agentes(c),agentes(c)),\\ asParaModificar(sinUno(conjtodo),c))
        ELSE asParaModificar(sinUno(conjtodo),c)
        FI
    }
    FI
}
\tadAxioma{placaDelAs(pos,conjplacas,dic)}{
    \IF Obtener(dameUno(conjplacas),dic).posicion $=$ pos THEN dameUno(conjplacas)
    ELSE placaDelAs(pos,sinUno(conjplacas),dic)
    FI
}
\tadAxioma{periferia(conj,c)}{
    \IF $\emptyset$?(conjAs) THEN $\emptyset$
    ELSE posContigua(dameUno(conj),grilla(c))\\ $\cup$ periferia(sinUno(conj),c)
    FI
}



\tadAxioma{hAtrapados(conjContiguo,c)}{ \IF $\emptyset$?(conjContiguo) THEN $\emptyset$
    ELSE{
        \IF dameUno(conjContiguo)$\in$hippies(c)
        THEN{
            \IF esAtrapadoH?(posContiguas(dameUno(conjContiguo),grilla(c)))
            THEN Ag(dameUno(conjContiguo),\\ hAtrapados(sinUno(conContiguo),c))
            ELSE hAtrapados(sinUno(conjContiguo),c)
            FI
        }
        ELSE hAtrapados(sinUno(conjContiguo),c)
        FI
    }
    FI
}

\tadAxioma{esAtrapadoH?(conjcont,c)}{
    \IF $\emptyset$?(conjclaves) THEN false
    ELSE{
        \IF hayPosVacia?(conjcont,c) THEN false
        ELSE{
            \IF dameUno(conjcont) $\in$\\ conjuntoPosAs(claves(agentes(c)),agentes(c)) THEN true
            ELSE esAtrapadoH?(sinUno(conjcont),c)
            FI
        }
        FI
    }
    FI
}

\tadAxioma{sancionarAs(conjAs,c)}{ \IF $\emptyset$?(conjAs) THEN agentes(c)
    ELSE definir(dameUno(conjAs),\\ (Obtener(dameUno(conjAs),agentes(c)).posicion,\\ Obtener(dameUno(conjAs),agentes(c)).capturas,\\  Obtener(dameUno(conjAs),agentes(c)).sanciones +1),\\ sancionarAs(sinUno(conjAs),c))
    FI
}
%false-> premia / true -> sanciona
\tadAxioma{premiarYSancionar(conjP,c)}{ \IF $\emptyset$?(conjP) THEN agentes(c)
    ELSE{
        \IF $\pi_2$(dameUno(conjP))$=$false
        THEN definir($\pi_1$(dameUno(conjP)),\\ (Obtener($\pi_1$dameUno(conjP),agentes(c).posicion,\\ Obtener($\pi_1$dameUno(conjP),agentes(c).capturas +1,\\ Obtener($\pi_1$dameUno(conjP),agentes(c)).sanciones),\\  premiarYSancionar(sinUno(conjP),c))
        ELSE definir(dameUno(conjP),\\ (Obtener($\pi_1$dameUno(conjP),agentes(c)).posicion,\\  Obtener($\pi_1$dameUno(conjP),agentes(c)).capturas,\\ Obtener($\pi_1$dameUno(conjP),agentes(c)).sanciones+1,\\  premiarYSancionar(sinUno(conjP),c))
        FI
    }
    FI
}
\tadAxioma{asSancPrem(conjPrem,conjSanc)}{
    \IF $\emptyset$?(conjPrem)$\land \emptyset$?(conjSanc) THEN $\emptyset$
    ELSE{
        \IF $\neg \emptyset$?(conjPrem)$\land \emptyset$?(conjSanc) 
        THEN Ag(\\ (dameUno(conjPrem),false), asSancPrem(sinUno(conjPrem),conjSanc))
        ELSE{
            \IF $\emptyset$?(conjPrem)$\land \neg \emptyset$?(conjSanc)
            THEN Ag(\\ (dameUno(conjSanc),true),\\ asSancPrem(conjPrem,sinUno(conjSanc)))
            ELSE{
                    Ag((dameUno(conjPrem),false), Ag((dameUno(conjSanc), true),\\ asSancPrem(sinUno(conjPrem),sinUno(conjSanc))))
            }
            FI
        }
        FI
    }
    FI
}
\tadAxioma{hayHippies?(c)}{\IF $\#$Hippies(c) $> 0$ THEN true ELSE false FI}
\tadAxioma{movEntradaCercana(c, p)}{
    \IF nuevaPosicion(posAS(c, p), posQueAcercan(c,posAS(c,p),\\ entradasCercanas(posAS(c, p),grilla(c)))= posAS(c, p)
    THEN agentes(c)
    ELSE movYActualizar(p,nuevaPosicion(posAS(c, p),\\ posQueAcercan(c,posAS(c,p),entradasCercanas(posAS(c,p), grilla(c))),c)
    FI
}
\tadAxioma{nuevaPosicion(posactual,conjpos,c)}{
    \IF $\emptyset$?(conjpos) THEN posactual
    ELSE {
    \IF posVacia?(dameUno(conjpos),c) THEN dameUno(conjpos)
        ELSE nuevaPosicion(posactual, sinUno(conjpos),c)
        FI  
    }
    FI
}
\tadAxioma{posAS(c, p)}{Obtener(p,agentes(c)).poscion}

\tadAxioma{posQueAcercan(c, pos, objetivos)}{
    \IF $\emptyset$?(objetivos) THEN $\emptyset$
    ELSE Ag(posQueAcercanAUno(c, pos, dameUno(objetivos)), \\posQueAcercan(pos, sinUno(objetivos)))
    FI
}
\tadAxioma{posQueAcercanAUno(c, pos, objetivo)}{
    iterarDirs(objetivo, posContiguas(pos, grilla(c)), distTaxi(pos, objetivo))
}
\tadAxioma{iterarDirs(objetivo, contiguas, dist)}{
    \IF $\emptyset$?(contiguas) THEN $\emptyset$
    ELSE {
        \IF distTaxi(objetivo, dameUno(contiguas)) < dist THEN Ag(dameUno(contiguas), \\iterarDirs(objetivo, sinUno(contiguas), dist))
        ELSE iterarDirs(objetivo, sinUno(contiguas), dist)
        FI
    } 
    FI
}
\tadAxioma{distTaxi(pos,posPers)}{ distancia(pos.x, posPers.x) + distancia(pos.y, posPers.y)}
\tadAxioma{distancia(n,m)}{\IF $m-n < 0$ THEN $-(m-n)$ ELSE $m-n$ FI}

\tadAxioma{entradasCercanas}{ \IF distancia(pos.y, alto(g)-1) $<$ distancia(pos.y, 0) 
    THEN Ag((pos.x, alto(g)-1) $\emptyset$)
    ELSE {
        \IF distancia(pos.y, 0) $<$ distancia(pos, alto(g)-1)
        THEN Ag((pos.x, 0) $\emptyset$)
        ELSE Ag((pos.x, alto(g)-1), Ag((pos.x, 0) $\emptyset$))
        FI
    }
    FI
}

\tadAxioma{movYActualizarAS(p,nuevapos,c)}{
    actualizarAs(nuevapos, definir(p,\\ (nuevapos, obtener(p,agentes(c)).capturas,\\ obtener(p,agentes(c)).sanciones), agentes(c)),c)
}

\tadAxioma{perseguirHippies(p,c)}{ 
    \IF nuevaPosicion(posAS(c,p), posQueAcercan(c, posAS(c,p),\\ personasCercanas(posAS(c,p), hippies(c))$=$posAS(c,p) THEN agentes(c)
    ELSE movYActualizarAS(p,\\ nuevaPosicion(posAS(c,p), posQueAcercan(c, posAS(c,p),\\ personasCercanas(posAS(c,p), hippies(c)),c)
    FI
}
\tadAxioma{personasCercanas(pos,conj)}{ \IF $\emptyset$?(conj) THEN $\emptyset$
    ELSE{
        \IF distTaxi(pos,dameUno(conj)) =\\ menorDistanciaAPos(pos,conj) THEN Ag(dameUno(conjH),personasCercanas(pos,sinUno(conj)))
        ELSE personasCercanas(pos,sinUno(conj))
        FI
    }
    FI
}
\tadAxioma{menorDistanciaAPos(pos,conj)}{ \IF $\#$(conj) =1 THEN distTaxi(pos,dameUno(conj))
    ELSE min(distTaxi(pos,dameUno(conj)),\\ menorDistanciaAPos(pos,conj))
    FI
}

\tadAxioma{seConvierteEnHippie?(c, pos)}{cantHippiesAdyacentes(c, pos) $\geq$ 2}
\tadAxioma{cantHippiesAdyacentes(c, pos)}{contarHippies(c, posContiguas(pos, grilla(c)))}
\tadAxioma{contarHippies(c, contiguas)}{
    \IF $\emptyset$?(contiguas) THEN 0
    ELSE{
        contarHippies(c, sinUno(contiguas)) + \\\IF esHippie?(c, dameUno(contiguas)) THEN 1 ELSE 0 FI
    }
    FI
}

\tadAxioma{estudiantesFalopeados(c, pos)}{buscarFalopeados(c, posContiguas(pos, grilla(c)))}
\tadAxioma{buscarFalopeados(c, contiguas)}{
    \IF $\emptyset$?(contiguas) THEN $\emptyset$
    ELSE {
        \IF seConvierteEnHippie(c, dameUno(contiguas)) THEN Ag(dameUno(contiguas), \\buscarFalopeados(c, sinUno(contiguas)))
        ELSE buscarFalopeados(c, sinUno(contiguas))
        FI
    }
    FI
}

\tadAxioma{nuevaPosHippie(c,pos)}{nuevaPosicion(pos, posQueAcercan(personasCercanas(pos, \\estudiantes(c))), c)}

\tadAxioma{hippiesConvertidos(c, pos)}{buscarConvertidos(c, posContiguas(pos, grilla(c)))}
\tadAxioma{buscarConvertidos(c, contiguas)}{
    \IF $\emptyset$?(contiguas) THEN $\emptyset$
    ELSE {
        \IF seConvierteEnEstu?(c, dameUno(contiguas)) THEN Ag(dameUno(contiguas), \\buscarConvertidos(c, sinUno(contiguas)))
        ELSE buscarConvertidos(c, sinUno(contiguas))
        FI
    }
    FI
}
\tadAxioma{seConvierteEnEstu?(c, pos)}{cantEstuAdyacentes(c, pos) = 4}
\tadAxioma{cantEstuAdyacentes(c, pos)}{contarEstudiantes(c, posContiguas(pos, grilla(c)))}
\tadAxioma{contarEstudiantes(c, contiguas)}{
    \IF $\emptyset$?(contiguas) THEN 0
    ELSE{
        contarEstudiantes(c, sinUno(contiguas)) + \\\IF esEstudiante?(c, dameUno(contiguas)) THEN 1 ELSE 0 FI
    }
    FI
}

\tadAxioma{hippiesCapturados(c, pos)}{buscarCapturados(c, posContiguas(pos, grilla(c)))}
\tadAxioma{buscarCapturados(c, contiguas)}{
    \IF $\emptyset$?(contiguas) THEN $\emptyset$
    ELSE {
        \IF esAtrapadoH?(c, posContiguas(dameUno(contiguas), grilla(c))) \\THEN Ag(dameUno(contiguas), \\buscarCapturados(c, sinUno(contiguas)))
        ELSE buscarCapturados(c, sinUno(contiguas))
        FI
    }
    FI
}
\tadAxioma{capturadoH(posH,c)}{ esAtrapadoH?(posContiguas(posH,grilla(c)),c)}

\tadAxioma{cantidadDeHippies(c)}{ $\#$ hippies(c)}

\tadAxioma{cantidadDeEstudiantes(c)}{ $\#$ estudiantes(c)}

\tadAxioma{asMasVigilante(c)}{elDeMenorPlaca(losMasVigilantes(agentes(c), \\numCapturasMax(conjuntoDeCapturas(agentes(c)))))}
\tadAxioma{elDeMenorPlaca(placas)}{\IF $\#$placas $= 1$
	THEN dameUno(placas)
	ELSE min(dameUno(placas), elDeMenorPlaca(sinUno(placas)))
	FI
}
\tadAlinearAxiomas{losMasVigilantes(agentes, numCapturas)}
\tadAxioma{losMasVigilantes(agentes, numCapturas)}{ \IF $\#$claves(agentes) $= 1$
    THEN dameUno(claves(agentes))
    ELSE{
        \IF capturas(dameUno(claves(agentes))) $=$ numCapturas THEN Ag(dameUno(claves(agentes)), \\losMasVigilantes(borrar(dameUno(claves(agentes)), \\agentes), numCapturas))
        ELSE losMasVigilantes(borrar(dameUno(claves(agentes)), \\agentes), numCapturas)
        FI
    }
    FI
} %hay que hacer la restriccion que diga que numeroDeCapturas es un numero perteneciente al conjunto de capturas de todas las claves del Dicc
\tadAlinearAxiomas{conjuntoDeCapturas(agentes)}
\tadAxioma{numCapturasMax(capturas)}{
    \IF $\#$capturas $= 1$ THEN dameUno(capturas)
    ELSE max(dameUno(capturas), numCapturasMax(sinUno(capturas))) 
    FI 
}
\tadAxioma{conjuntoDeCapturas(agentes)}{
    \IF $\#$ claves(agentes) $= 1$ THEN obtener(dameUno(claves(agentes)),agentes).capturas 
    ELSE Ag(obtener(dameUno(claves(agentes))).capturas, \\conjuntoDeCapturas(borrar(dameUno(claves(agentes))), agentes)) 
    FI
}

\tadAxioma{nuevaPosAS(c, placa)}{\IF sanciones((Obtener(placa,agentes(c)))) $\geq3$
    THEN posicion(obtener(placa,agentes(c)))
    ELSE {
        \- \IF $\neg$hayHippies?(c) THEN posicion(obtener(placa, movEntradaMasCercana(c, placa)))
        ELSE posicion(obtener(placa, perseguirHippiesCercanos(placa,c)))
        FI
    }
    FI
}


\end{tad}


\section{TAD \tadNombre{Grilla}}

\begin{tad}{\tadNombre{Grilla}}

\tadGeneros{grilla}

\tadExporta{grilla, generadores, observadores basicos}

\tadIgualdadObservacional{g1}{g2}{grilla}{$(ancho(g1) \igobs ancho(g2))\ \land$\\ $(alto(g1) \igobs alto(g2))\ \land$\\ $(obstaculos(g1) \igobs obstaculos(g2))$}

\tadAlinearFunciones{agregarObstaculo}{grilla, posicion}{}

% Generadores
\tadGeneradores

\tadOperacion{crearGrilla}{nat, nat}{grilla}{}
\tadOperacion{agregarObstaculo}{grilla, posicion}{grilla}{}

% Observadores
\tadObservadores

\tadOperacion{ancho}{grilla}{nat}{}
\tadOperacion{alto}{grilla}{nat}{}
\tadOperacion{obstaculos}{grilla}{conj(posicion)}{}

% Axiomas
\tadAxiomas[\paratodo{nat}{an, al}, \paratodo{grilla}{g}, \paratodo{posicion}{p}]
\tadAlinearAxiomas{obstaculos(agregarObstaculo(g, p))}

\tadAxioma{ancho(crearGrilla(an, al))}{an}
\tadAxioma{ancho(agregarObstaculo(g, p))}{ancho(g)}

\tadAxioma{alto(crearGrilla(an, al))}{al}
\tadAxioma{alto(agregarObstaculo(g, p))}{alto(g)}

\tadAxioma{obstaculos(crearGrilla(an, al))}{$\emptyset$}
\tadAxioma{obstaculos(agregarObstaculo(g, p))}{Ag(p, obstaculos(g))}

\end{tad}

\section{Notas}

\begin{itemize}

\item nuevaPosicion toma el conjunto de posiciones a las que deberia moverse y lo va iterando hasta que encuentra una en la que no hay nada y la devuelve. Si el conjunto de posiciones que toma tiene todas las posiciones ocupadas devuelve la que esta (no se mueve).
\item movYActualizarAS hace mover al as a la nueva posicion (si es que la hay, se queda con la que tiene), y le pasa esa posicion, el diccionario definiendo al as con esa nueva posicion y el campus a actualizarAs.
\item actualizarAs ya me toma al diccionario con el as en la nueva posicion y revisa si hay hippies o estudiantes atrapados, si hay hippies atrapados premia a los as que los atraparon, si hay estudiantes atrapados sanciona a los as que los atraparon.
\item periferia: devuelve todas las posiciones de al rededor de cada una de las posiciones del conjunto que toma, sirve para ver que tiene al rededor el estudiante o el hippie atrapado.
\item asParaModificar: agarra el conjunto que devuelve periferia y busca los as a los que habria que agregarle una sancion o una captura
\item distTaxi: devuelve la distancia a la que esta una persona de otra en el campus teniendo en cuenta la geometria del taxista
\item premiarYSancionar: es el caso en que mover un as hace que queden atrapados hippies y estudiantes y se tenga que sancionar y premiar a la vez. Lo que hicimos es que tome un multiconjunto de tuplas de dos posiciones (placa, bool) si el bool es false se premia al as, sino se sanciona a ese as.
\item asSancPrem: lo que hace esta funcion es tomar un multiconjunto de posiciones de los as que hay que premiar y otro multiconjunto de posiciones de as para los que hay que sancionar y por cada uno de los elementos lo mete en una tupla con el bool respectivo de premiar y sancionar (si es para sancionar agrega un true en la tupla, si es para premiar agrega un false) y con la placa del as.
\item Cuando un estudiante entra no puede ser atrapado por un AS ya que tiene libertad de movimiento. Puede salir por la entrada.
\item estudiantesFalopeados devuelve un conunto de los estudiantes que fueron convertidos al hippismo por concecuencia de que un hippie apareciera en la posicion pasada como parametro
\item hippiesConvertidos devuelve un conunto de los hippies que fueron convertidos en estudiante por concecuencia de que un estudiante apareciera en la posicion pasada como parametro
\item hippiesCapturados devuelve un conunto de los hippies que fueron atrapados por concecuencia de que algo se mueva o aparezca en la posicion pasada como parametro
\item posQueAcercan toma una posicion, que es donde estoy parado, y un conjunto de posiciones, a las cuales voy a querer acercarme. Lo que devuelve es un conjunto de posiciones contiguas que se encuentran a menor distancia del objetivo que la posicion en la que estoy parado

\end{itemize}
\end{document}
